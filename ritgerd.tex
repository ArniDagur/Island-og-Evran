% Header {{{
% vim: ft=tex:fdm=marker
\documentclass[a4paper]{article}

% Plugins
\usepackage[T1]{fontenc}
\usepackage[utf8]{inputenc}
\usepackage[icelandic]{babel}

\usepackage{url}

\title{Krónan eða Evran fyrir Ísland?}
\author{Árni Dagur}
% }}}
\begin{document}
\maketitle

\section{Inngangur}

Þó tal um upptöku gjaldmiðils annars en krónunnar hafi lengi verið til staðar, magnaðist sú umræða mjög eftir hrunið mikla árið 2008.(?)

Í viðtali við \textit{The Guardian} lýsti háttsettur Íslenskur embættismaður sem ekki vildi láta nafns síns getið vantrausti sínu til krónunnar. \textit{''The Krona is dead. We need a new currency. The only serious option is the euro.''} sagði hann.\cite{traynor_2009}

Árið 2004 gerði Þórarinn G. Pétursson, sem nú er aðalhagfræðingur Seðlabanka Íslands, rannsókn í samvinnu við breskan hagfræðing þar sem reynt var að leggja tölu á kosti þess að ganga inn í Evrusvæðið. Þeir komust að þeirri niðurstöðu að það megi búast við 60\% viðskiptaaukningu við það að ganga í Evrusvæðið. Helmingur þess gróða mætti rekja til þess eins að ganga í ESB, en hinn helminginn til upptöku Evrunnar.\cite{icb_wp_26}

\section{Vandamál}

\subsection{Dæmið um Grikkland}

Í venjulegum kringumstæðum þar sem land hefur hefur fullt vald yfir gjaldmiðli sínum mun það bregðast við miklu atvinnuleysi með því að prenta meiri pening. Það lækkar virði gjaldmiðilsins og þar með hækkar ferðamannastraum, fjárfestingar, og gróða útflutningsgreina. Ef atvinnuleysi er lágt, aftur á móti, mun það prenta minni pening og þar með hækka kaupmátt og lækka verð á vörum (Klein, 2015). Þetta getur þó Grikkland ekki gert af því hvernig Evrusvæðið er sett upp. Þó að Grikkland stjórni sínum eigin ríkisfjármálum, stjórnar Seðlabanki Evrópusambandsins  peningastefnu þess (Pogorelec). Styrkur gjaldmiðils sem er tilvalinn fyrir Þýskaland, þar sem atvinnuleysi er 4.2\% (The European Union) er allt annar en styrkur gjaldmiðils sem er tilvalinn fyrir Grikkland, þar sem atvinnuleysi er 23.2\% (The European Union).

\bibliography{bibliography}
\bibliographystyle{plain}

\end{document}
