% Boilerplate {{{
% vim: ft=tex:fdm=marker
\documentclass{article}
\usepackage[T1]{fontenc}
\usepackage[utf8]{inputenc}
\usepackage[icelandic]{babel}
\title{Ísland og evran}
\author{Árni Dagur}
% }}}

\begin{document}
\maketitle

Í venjulegum kringumstæðum þar sem land hefur hefur fullt vald yfir gjaldmiðli sínum mun það bregðast við miklu atvinnuleysi með því að prenta meiri pening. Það lækkar virði gjaldmiðilsins og þar með hækkar ferðamannastraum, fjárfestingar, og gróða útflutningsgreina. Ef atvinnuleysi er lágt, aftur á móti, mun það prenta minni pening og þar með hækka kaupmátt og lækka verð á vörum (Klein, 2015). Þetta getur þó Grikkland ekki gert af því hvernig Evrusvæðið er sett upp. Þó að Grikkland stjórni sínum eigin ríkisfjármálum, stjórnar Seðlabanki Evrópusambandsins  peningastefnu þess (Pogorelec). Styrkur gjaldmiðils sem er tilvalinn fyrir Þýskaland, þar sem atvinnuleysi er 4.2\% (The European Union) er allt annar en styrkur gjaldmiðils sem er tilvalinn fyrir Grikkland, þar sem atvinnuleysi er 23.2\% (The European Union).

Hello world!

\end{document}
